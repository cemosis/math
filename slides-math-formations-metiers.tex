% -*- coding: utf-8
% /solutions/conference-talks/conference-ornate-20min.fr.tex, 22/02/2006 De Sousa

\documentclass[slideopt,A4]{beamer}

\usetheme{default}

\useoutertheme{infolines}

\usepackage[]{fontspec}
\usepackage[T1]{fontenc}
\usepackage[type1]{libertine}
\renewcommand{\ttdefault}{cmtt}

\usepackage{lastpage}
\usepackage{mdframed}

\usepackage{anyfontsize}
\setmainfont{linuxbiolinumo}
\setsansfont{linuxbiolinumo}

\definecolor{fontmaincolor}{HTML}{}
\definecolor{bgcolor}{HTML}{FFFFFF}
\definecolor{bgshadecolor}{HTML}{DDDDDD}
\graphicspath{%
  {./Figures/}%
}
%\usepackage [T1] {inputenc}
%\usepackage [utf8] {inputenc}

\usepackage{eso-pic}
\usepackage{monster2e}
\usepackage{tikz}
\usepackage{layout}
\usepackage{xcolor}
\usepackage{amsmath,amssymb,wasysym}
\usepackage{fancybox}
\usepackage{graphicx}
\usepackage{multimedia}
\usepackage[absolute,showboxes,overlay]{textpos}
\TPshowboxesfalse
\textblockorigin{0mm}{0mm}

\definecolor{rouge1}{RGB}{226,0,38}  % red P
\definecolor{orange1}{RGB}{243,154,38}  % orange P
\definecolor{jaune}{RGB}{254,205,27}  % jaune P
\definecolor{blanc}{RGB}{255,255,255} % blanc P

\definecolor{rouge2}{RGB}{230,68,57}  % red S
\definecolor{orange2}{RGB}{236,117,40}  % orange S
\definecolor{taupe}{RGB}{134,113,127} % taupe S
\definecolor{gris}{RGB}{91,94,111} % gris S
\definecolor{bleu1}{RGB}{38,109,131} % bleu S
\definecolor{bleu2}{RGB}{28,70,114} % bleu S
\definecolor{vert1}{RGB}{133,146,66} % vert S
\definecolor{vert2}{RGB}{157,193,7} % vert S

\definecolor{rouge3}{RGB}{255,200,130}  

\setbeamertemplate{navigation symbols}{}
\setbeamertemplate{blocks}[rounded][shadow=false]
\setbeamercolor{block title}{fg=white, bg=rouge1}
\setbeamercolor{block body}{fg=black, bg=rouge3}

\renewcommand{\sfdefault}{lmss}
\sffamily

\setbeamersize{text margin left=1cm,text margin right=1cm}


\AtBeginSection[] {
  \begin{frame}<beamer>{Plan}
   \tableofcontents[sectionstyle=show/shaded,subsectionstyle=show/shaded/hide]
%    \tableofcontents[currentsection]
  \end{frame}
 }

\AtBeginSubsection[] {
  \begin{frame}<beamer>{Plan}
   \tableofcontents[sectionstyle=show/shaded,subsectionstyle=show/shaded/hide]
%    \tableofcontents[currentsection,currentssubection]
  \end{frame}
 }


\newcommand{\ds}{\displaystyle}
\newcommand{\dg}{^\circ}
\newcommand{\saut}{\vspace*{3mm}\\}
\newcommand{\RR}{\hbox{\bf I\hspace*{-1mm}R}}

\begin{document}


\setbeamertemplate{background canvas}[vertical shading][top=white,middle=white,bottom=white]

\setbeamertemplate{footline}{ \hspace{5em} \textcolor{white} {Les Mathématiques}\hspace{2em}\null \vspace*{3pt}}


\begin{frame}

\begin{textblock*}{15cm}(13mm,30mm)
{\textcolor{red} {
{\huge\bf Mathématiques appliquées: }\\[2mm]
{\huge\bf  des maths en lien avec la société}\\[2mm]
{\huge\bf   et son environnement}\\[8mm] }}
%{\huge\bf  }\\[12mm]
{\textcolor{black} {
   	{\Large Philippe Helluy \& Christophe Prud'homme}\\[2mm]
	{\Large Université de Strasbourg}\\[2mm]
 
 \includegraphics[height=1.2cm]{LOGOS/logoCemosis}	\hspace*{40mm}	
 \includegraphics[height=1.2cm]{LOGOS/logoUDS} 
	\\
	}
	}
	\end{textblock*}

\end{frame}


%%%%%%%%%%%%%%%%%%%%%%%%%%%

\setbeamertemplate{background canvas}[vertical shading][top=blanc, middle=blanc,bottom=blanc]

\setbeamercolor{toto}{fg=white,bg=rouge1}

\setbeamertemplate{footline}
{
\begin{beamercolorbox}[wd=1\paperwidth,ht=15.5pt]{toto}
\hspace{-1.6mm}	
  \raisebox{1.2ex}
  {  \includegraphics[height=.6cm]{LOGOS/logoCemosis}}
  \raisebox{2.5ex}
 { P. Helluy \& C. Prud'homme (UNISTRA/CEMOSIS)  }
\hspace{\fill}	
 % \raisebox{1.2ex}
%\hspace{\fill}	
%  \raisebox{2.5ex}
%  { \insertframenumber \hspace{1mm}  }
  \raisebox{2.5ex}
 { irma/moco }
  \raisebox{1ex}
{  \includegraphics[height=.6cm]{LOGOS/logoUDS}}
\end{beamercolorbox} 
}

%
%
%%%%%%%%%%%%%%%%%%%%%%%%%%%%%%%%%%%%%%

\begin{frame}{{\Large Plan}}
  \tableofcontents
  % Vous pouvez, si vous le souhaiter ajouter l'option [pausesections]
\end{frame}



%\begin{frame}
%Crypto: ajouter exemple de codage / décodage  eco spé maths TS\\[10mm]
%
%Simul: faire la discrétisation d'une équation de transport: 
%- on prend 2 boites, on écrit les quantités qui passent de l'une à l'autre en un pas de temps delta t
%- puis on en prend 3, etc...
%
%\end{frame}

%%%%%%%%%%%

\section{Mathématiques appliquées et métiers des maths}



\begin{frame}
\frametitle{Pour commencer...}

%\textcolor{black} 
%
%
%\begin{minipage}{70mm}
\begin{itemize}
\item {\large A quoi servent les mathématiques ?}
\item  {\large Y a-t-il des métiers après des études en mathématiques ?}
\end{itemize}
%\end{minipage}
%
\begin{center}
\includegraphics[width=0.20\linewidth]{maths2}$\quad$
\includegraphics[width=0.27\linewidth]{maths1}$\quad$
\includegraphics[width=0.24\linewidth]{einstein}
\end{center}

%
\end{frame}



%%%%%%%%%%%%%%%%%%%%%



\begin{frame}
\frametitle{Les mathématiques sont omniprésentes dans notre société}
%
%Les mathématiques sont omniprésentes dans notre société.
\begin{center}
\includegraphics[width=0.90\linewidth]{mix.png}
\end{center}

\end{frame}
%
%

\begin{frame}
\frametitle{Créer des images...}
%
\begin{center}
\includegraphics[width=0.40\linewidth]{LaraCroft.png}$\,$
\includegraphics[width=0.30\linewidth]{avatar}\\
\includegraphics[width=0.50\linewidth]{tintin}
\end{center}
%
\end{frame}
%
%
\begin{frame}
\frametitle{Créer des images...}
%
\begin{center}
\includegraphics[width=0.40\linewidth]{tete.png}$\quad$
\includegraphics[width=0.40\linewidth]{tete2.png}
\end{center}
%
\end{frame}
%


%
\begin{frame}
\frametitle{Compresser les images et les sons: MP3, MP4, jpeg, mov...}
%
\begin{center}
\begin{tabular}{cc}
\textcolor{red}{Image originale} & \textcolor{red}{Image compressée} \\
\includegraphics[width=0.40\linewidth]{lana-original.png} &
\includegraphics[width=0.40\linewidth]{lana-compresse.png}
\end{tabular}
\end{center}
%
\pause
%
\begin{center}
~\\[-10mm]
\includegraphics[width=0.4\linewidth]{ipod.jpg}
\end{center}
%
\end{frame}
%

%
\begin{frame}
\frametitle{Compresser les images et les sons: MP3, MP4, jpeg, mov...}
%
\includegraphics[width=0.3\linewidth]{fourier} $\qquad$
\includegraphics[width=0.5\linewidth, angle=-1.5]{fourier4} \\[5mm]
\centerline{\includegraphics[width=0.6\linewidth]{fourier2}}
%
\end{frame}
%

%
\begin{frame}
\frametitle{Imagerie médicale...}
\small
%
%
\begin{block}{Principe}
On soumet le corps à un signal (magnétique, électrique, acoustique, ...), et en fonction de la réponse obtenue, on reconstruit  une image de la zone observée (problème inverse).
\end{block}
%
\centerline{\includegraphics[width=0.7\linewidth]{imagerie-medicale.png}}
%
\end{frame}
%


%
\begin{frame}
\frametitle{Statistiques...}
%
%
\includegraphics[width=0.3\linewidth]{medicaments.jpg}$\;$
\includegraphics[width=0.3\linewidth]{banque}$\;$
\includegraphics[width=0.3\linewidth]{assurance}
%
\end{frame}
%


%
\begin{frame}
\frametitle{Crypter...}
%
% 
\begin{minipage}{5cm}
~\\[-50mm]
\includegraphics[width=0.75\linewidth]{carte-bleue.jpg}\\
\includegraphics[width=0.95\linewidth]{paypal.jpg}
\end{minipage}
\includegraphics[width=0.5\linewidth]{reseau}
%
\end{frame}
%



%
\begin{frame}
\frametitle{Modéliser...}
%
%
\begin{minipage}{35mm}
\includegraphics[width=0.99\linewidth]{edf.png}
\end{minipage}$\;$
\begin{minipage}{70mm}
\includegraphics[width=0.99\linewidth]{edf2.png}\\[10mm]
\centerline{\alert{EDF}}
\end{minipage}
%
\end{frame}
%

%
\begin{frame}
\frametitle{Modéliser...}
%
Stockage de déchets radioactifs $\rightarrow$ Modélisation des écoulements en milieux poreux autour du site\\
\centerline{\includegraphics[width=0.8\linewidth]{stockage.png}}
%
\end{frame}
%

%
\begin{frame}
\frametitle{Modéliser...}
%
Contrôle non destructif
%
\centerline{\includegraphics[width=0.75\linewidth]{controle-nondestructif.png}}
%
\end{frame}
%


%
\begin{frame}
\frametitle{Modéliser...}
%
Fusion nucléaire contrôlée : plasma confiné dans la chambre magnétique (ITER)
%
\centerline{\includegraphics[width=0.8\linewidth]{iter.png}}
%
\end{frame}
%




%
\begin{frame}
\frametitle{Modéliser...}
%
% 

\includegraphics[width=0.4\linewidth]{voiture.png}$\qquad$
\includegraphics[width=0.5\linewidth]{pneu.png}\\[3mm]
\centerline{\includegraphics[width=0.5\linewidth]{avion}}
%
\end{frame}
%

%
\begin{frame}
\frametitle{Modéliser...}
%
% 
\centerline{
\includegraphics[width=0.2\linewidth]{chaise.png}$\qquad$
\includegraphics[width=0.2\linewidth]{coca}$\qquad$
\includegraphics[width=0.2\linewidth]{lessive} }
%
\end{frame}
%


%
\begin{frame}
\frametitle{Il y a des dizaines de "métiers des maths" {\small (pas seulement prof !) }}
%
\begin{center}
\includegraphics[width=0.4\linewidth]{onisep.pdf}
\end{center}
%
\centerline{\htmladdnormallink{Quels Métiers pour les matheux? (ONISEP!)}{http://www.onisep.fr/Toute-l-actualite-nationale/Decouvrir-les-metiers/Decembre-2014/Quels-metiers-pour-les-matheux}}
%
\end{frame}


\end{document}

