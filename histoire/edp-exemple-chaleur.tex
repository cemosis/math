
\begin{frame} \frametitle{Équation de la chaleur}
  \only<1>{
  $$
  \begin{aligned}
    \frac{\partial u}{\partial t} + c \cdot \nabla u - \nabla \cdot( a \nabla u ) = f & \mbox{ dans } \Omega\\
    u = u_b  & \mbox{ sur } \Gamma_1\\
    -a \nabla u \cdot N = g & \mbox{ sur } \Gamma_2\\
    -a \nabla u \cdot N + \kappa (u - u_b) = 0 & \mbox{ sur } \Gamma_3
  \end{aligned}$$
}
  \only<2>{

  \begin{columns}
    \begin{column}{.5\linewidth}
      \begin{figure}[H]
        \centering
        \centerline{\includegraphics[width=\figwidth\textwidth]{figures/jpegs/prudhomme/labra_thumb2}}
        \caption{Circulation océanique dans la mer du Labrador}
        \label{fig:2}
      \end{figure}
    \end{column}
    \begin{column}{.5\linewidth}
      \begin{itemize}
      \item  Champ de température à 180 m. de profondeur, obtenu gr\^ace à un
        zoom au 1/15\textdegree dans un modèle au 1/3\textdegree de l'Atlantique Nord
      \item Phénomènes de Convection(par les courants marins) et
        Diffusion
      \item Acquisition/Assimilation de données
      \item Simulation complexe
      \item Nécessite des ressources de calcul importantes
      \end{itemize}
    \end{column}
  \end{columns}

  }

\end{frame}

