\begin{frame}
  \frametitle{Cycle de la simulation numérique}

  \tikzstyle{root concept}+=[concept color=white!80]
  \tikzstyle{level 1 concept}+=[concept color=red!80, sibling angle=72]
  \tikzstyle{every annotation}=[fill=black!50,opacity=0.5,text=white,scale=.7]
  \begin{tikzpicture}[->]
     \path[mindmap,concept color=black!60,text=white]
     node[concept,scale=.7] {Numerical Simulation Cycle}
     %node[concept,scale=.7] {Scientific Computing}
     [clockwise from=0]
     %child[concept,scale=.6] { node[concept,scale=.7] (phys) {Physics Mechanics Biology Processing} }
     child[concept,scale=.6] { node[concept,scale=.7] (phys) {Modélisation} }
     child[concept,scale=.6] { node[concept,scale=.7] (am) {Math. Appli} }
     child[concept,scale=.6] { node[concept,scale=.7] (nm) {Math. Num} }
     child[concept,scale=.6] { node[concept,scale=.7] (cs) {Informatique} }
     child[concept,scale=.6] { node[concept,scale=.7] (va) {Validation} }
     ;
     \node [annotation,below] at (phys.south east)
     {
       \begin{itemize}
       \item Santé
       \item Geophysique
       \item Astrophysique
       \item Prévision du temps
       \item Climat
       \item Physique des Plasma
       \item Aerodynamique
       \item Hydrodynamique
       \item MHD
       \item Rheologie
       \item Traitement des matériaux
       \item Fusion des métaux
       \item Finance
       \end{itemize}
     };
     \node [annotation,below] at (am.mid)
     {
       \begin{itemize}
       \item Statistiques
       \item Analyse fonctionnelle
       \item Equations aux dérivées partielles,
       \item Equations Stochastiques, etc.
       \end{itemize}
     };
     \node [annotation,below] at (nm.west)
    {
      \begin{itemize}
      \item Analyse numérique:
        Convergence, Erreurs
      \item Methods d'approximation:
        Discretization espace/time
        DF, EF, VF,  methodes
        spectrales, simulation de particules
      \item Algorithmes: complexité, précision
      \item Generation de Maillage , CAO
      \item Résolution de systemes (non)linéaires
      \end{itemize}
    };
    \node [annotation,above] at (cs.north)
    {
    \begin{itemize}
    \item Architecture: vectorielle, parallele,  scalaire, cluster.
    \item Systemes, Compilateurs. Librairies
      \item Data management, Visualisation
      \item Parallelisation: MPI, OpenMP
      \item Optimisation, Parameterisation
      \end{itemize}
    };
    \node [annotation,above] at (va.north)
    {
      \begin{itemize}
      \item Interpretation des résulats numériques
      \item Comparaison avec l'experience

      \item \alert{Correction des modèles}
      \end{itemize}
    }
    ;

    \begin{pgfonlayer}{background}
      \draw [concept connection]
      (phys) edge node[above,sloped]{Analyser} (am)
      (am) edge node[above,sloped]{Analyser} (nm)
      (nm) edge node[above,sloped]{Implementer} (cs)
      (cs) edge node[above,sloped]{Executer} (va)
      (va) edge node[above,sloped]{Corriger} (phys);
     \end{pgfonlayer}

\end{tikzpicture}
\end{frame}
