


% \begin{frame}
%   \frametitle{Les Formations de Master à Strasbourg}
%   \begin{block}{Master mention Mathématiques et Applications}
%     \begin{itemize}
%     \item \htmladdnormallink{Spécialité Calcul Scientifique et Mathématiques de
%       l'Information (CSMI)}{http://mathinfo.unistra.fr/offre-de-formation/master-mention-mathematiques-et-applications/csmi/}
%     \item \htmladdnormallink{Spécialité Mathématiques Fondamentales, parcours Recherche}{http://mathinfo.unistra.fr/offre-de-formation/master-mention-mathematiques-et-applications/maths-fondamentales-recherche/}
%     \item \htmladdnormallink{Spécialité Mathématiques Fondamentales, parcours Magistère de Mathématiques}{http://mathinfo.unistra.fr/offre-de-formation/master-mention-mathematiques-et-applications/maths-fondamentales-magistere/}
%     \item \htmladdnormallink{Spécialité Statistique, parcours Biostatistique et Statistiques Industrielles}{http://mathinfo.unistra.fr/offre-de-formation/master-mention-mathematiques-et-applications/statistique-bio-indus/}
%     \item \htmladdnormallink{Spécialité Statistique, parcours Actuariat}{http://mathinfo.unistra.fr/offre-de-formation/master-mention-mathematiques-et-applications/statistique-actuariat/}
%     \item \htmladdnormallink{Agrégation}{http://mathinfo.unistra.fr/offre-de-formation/master-mention-mathematiques-et-applications/master-agregation/}
%     \item \htmladdnormallink{CAPES}{http://mathinfo.unistra.fr/offre-de-formation/master-mention-mathematiques-et-applications/capes/}
%     \end{itemize}  
%   \end{block}
% \end{frame}

\begin{frame}
  \frametitle{CSMI}
  \begin{block}{Un master pour qui ?}
    Le Master est fait pour les étudiants de Licence désirant faire
    \begin{itemize}
    \item une carrière d'ingénieur
    \item une carrière de chercheur
    \item une carrière d'enseignant-chercheur
    \end{itemize}
  \end{block}
  
  
  \begin{block}{Objectif}
  Double compétence en Calcul Scientifique et Mathématiques de l'information
  \end{block}

  \begin{block}{Poursuite d'études}
    Doctorat en mathématiques appliquées
  \end{block}
\end{frame}

\begin{frame}
  \frametitle{Pré-requis CSMI}
  \begin{block}{}
    Bonnes bases en analyse, en algèbre et en informatique. 
    Goût pour la programmation, l'algorithmique et les applications des mathématiques.
  \end{block}
  \centerline{Une suggestion de parcours en Licence menant à CSMI}\\[-3mm]
  
  \begin{columns}[t]
    \column{.25\linewidth}
    \begin{block}{L1}
      \footnotesize\begin{itemize}
      \item Culture et pratique de l'informatique
      \item Electricité
      \item Modèles de calcul
      \end{itemize}
    \end{block}
    \column{.25\linewidth}
    \begin{block}{L2}
      \footnotesize\begin{itemize}
      \item Calcul scientifique
      \item Mécanique
      \item Équations différentielles
      \end{itemize}
    \end{block}
    \column{.25\linewidth}
    \begin{block}{L3S5}
      \footnotesize\begin{itemize}
       \item Anglais scientifique
       \item Algorithmique et structures de données 
       \item Calcul différentiel et intégral
       \item Statistique: étude de cas
       \end{itemize}
     \end{block}
     \column{.25\linewidth}
     \begin{block}{L3S6}
       \footnotesize\begin{itemize}
       \item Techniques d'analyse numérique
       \item Programmation orientée objet
       \item Équations différentielles
      \end{itemize}
     \end{block}
  \end{columns}

\end{frame}

\begin{frame}[allowframebreaks]{CSMI}
  \textbf{Contenu}\\
  Un savant mélange de théorie, numérique et informatique pour
  affronter les défis d'aujourd'hui et de demain !\\

  \textbf{Équipes et Structures de recherche associées}\\
  Équipe MOCO (IRMA) : Modélisation et Contrôle\\
  Équipe ICPS (ICUBE) : calcul parallèle\\
  Équipe AGA (IRMA) : arithmétique et géométrie algébrique\\
  Cemosis: Centre de Modélisation et Simulation de Strasbourg (\url{www.cemosis.fr})\\
  
  \framebreak

  \textbf{Entreprises Partenaires}\\
  Axessim, Gazomat, Holo3, Hager, Airbus, EDF, Plastic Omnium,
  Sigmaphi, Kitware \ldots

  \textbf{Stages}\\
  De nombreux stages à Strasbourg et en France (2 mois en M1, 6 mois
  en M2 en entreprise et/ou laboratoire de recherche)\\

\textbf{Échanges internationaux}\\
Possibilités d'échanges erasmus via le réseau ECMI (European
Consortium For Mathematics in Industry) un peu partout en Europe !\\
\url{www.ecmi-indmath.org/}
\end{frame}

\begin{frame}{Informations \& Contact}
  Sites web
  \begin{itemize}
  \item \url{http://mathinfo.unistra.fr/offre-de-formation/master-mention-mathematiques-et-applications/csmi/}
  \item \url{www.cemosis.fr/formation/csmi}
  \item \url{www.cemosis.fr/}
  \end{itemize}
  Contact
  \begin{itemize}
  \item Christophe Prud'homme \url{prudhomme@unistra.fr}
  \end{itemize}
\end{frame}
%%% Local Variables: 
%%% mode: latex
%%% TeX-master: "slides-math-formations-metiers"
%%% End: 
