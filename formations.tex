


\begin{frame}
  \frametitle{Les Formations de Master à Strasbourg}
  \begin{block}{Master mention Mathématiques et Applications}
    \begin{itemize}
    \item \htmladdnormallink{Spécialité Calcul Scientifique et Mathématiques de
      l'Information (CSMI)}{http://mathinfo.unistra.fr/offre-de-formation/master-mention-mathematiques-et-applications/csmi/}
    \item \htmladdnormallink{Spécialité Mathématiques Fondamentales, parcours Recherche}{http://mathinfo.unistra.fr/offre-de-formation/master-mention-mathematiques-et-applications/maths-fondamentales-recherche/}
    \item \htmladdnormallink{Spécialité Mathématiques Fondamentales, parcours Magistère de Mathématiques}{http://mathinfo.unistra.fr/offre-de-formation/master-mention-mathematiques-et-applications/maths-fondamentales-magistere/}
    \item \htmladdnormallink{Spécialité Statistique, parcours Biostatistique et Statistiques Industrielles}{http://mathinfo.unistra.fr/offre-de-formation/master-mention-mathematiques-et-applications/statistique-bio-indus/}
    \item \htmladdnormallink{Spécialité Statistique, parcours Actuariat}{http://mathinfo.unistra.fr/offre-de-formation/master-mention-mathematiques-et-applications/statistique-actuariat/}
    \item \htmladdnormallink{Agrégation}{http://mathinfo.unistra.fr/offre-de-formation/master-mention-mathematiques-et-applications/master-agregation/}
    \item \htmladdnormallink{CAPES}{http://mathinfo.unistra.fr/offre-de-formation/master-mention-mathematiques-et-applications/capes/}
    \end{itemize}  
  \end{block}
\end{frame}

\begin{frame}
  \frametitle{CSMI}
  \begin{block}{Un master pour qui ?}
    Le Master est fait pour les étudiants de Licence désirant faire
    \begin{itemize}
    \item une carrière d'ingénieur
    \item une carrière de chercheur
    \item une carrière d'enseignant-chercheur
    \end{itemize}
  \end{block}
  
  
  \begin{block}{Objectif}
  Double compétence en Calcul Scientifique et Mathématiques de l'information
  \end{block}

\end{frame}

\begin{frame}
  \frametitle{Pré-requis CSMI}
  \begin{block}{}
    Bonnes bases en analyse, en algèbre et en informatique. 
    Goût pour la programmation, l'algorithmique et les applications des mathématiques.
  \end{block}
  \centerline{Une suggestion de parcours en Licence menant à CSMI}\\[-3mm]
  
  \begin{columns}[t]
    \column{.25\linewidth}
    \begin{block}{L1}
      \footnotesize\begin{itemize}
      \item Culture et pratique de l'informatique
      \item Electricité
      \item Modèles de calcul
      \end{itemize}
    \end{block}
    \column{.25\linewidth}
    \begin{block}{L2}
      \footnotesize\begin{itemize}
      \item Calcul scientifique
      \item Mécanique
      \item Équations différentielles
      \end{itemize}
    \end{block}
    \column{.25\linewidth}
    \begin{block}{L3S5}
      \footnotesize\begin{itemize}
       \item Anglais scientifique
       \item Algorithmique et structures de données 
       \item Calcul différentiel et intégral
       \item Statistique: étude de cas
       \end{itemize}
     \end{block}
     \column{.25\linewidth}
     \begin{block}{L3S6}
       \footnotesize\begin{itemize}
       \item Techniques d'analyse numérique
       \item Programmation orientée objet
       \item Équations différentielles
      \end{itemize}
     \end{block}
  \end{columns}

\end{frame}
%%% Local Variables: 
%%% mode: latex
%%% TeX-master: "slides-math-formations-metiers"
%%% End: 
